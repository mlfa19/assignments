\documentclass{tufte-handout}

\usepackage{../CommonLatexPackages/machine_learning_preamble_1.0}
\fancypagestyle{firstpage}

{\rhead{Notation Conventions \linebreak \textit{Version: \today}}}

\title{Notation Conventions}
\author{Machine Learning}
\date{Fall 2019}

\begin{document}

\maketitle
\thispagestyle{firstpage}

\section{Notation}

\subsection{Scalars}
We will use lower-case, unbolded letters to refer to scalar quantities.  For example, we would refer to the scalar quantity x using the following notation.
\begin{align}
x
\end{align}

\subsection{Vectors}

We will use lower-case, bolded letters to refer to vector quantities.  For example, we would refer to the vector quantity v using the following notation.

\begin{align}
\mlvec{v}
\end{align}

\paragraph{Vector Indexing}
We will use the notation $v_i$ to refer to the $i$th element of the vector $\mlvec{v}$.

\paragraph{Row Versus Column Vectors}
When we talk about a vector, we will \emph{always} be referring to a column vector (i.e., a matrix with shape $d \times 1$).


\subsection{Matrices}

We will use upper-case, bolded letters to refer to matrix quantities.  For example, we would refer to the matrix quantity A using the following notation.

\begin{align}
\mlmat{A}
\end{align}

\paragraph{Matrix Indexing}
\be
\item We will refer to the $i$th column of the matrix $\mlmat{A}$ as $\mlvec{a}_{i}$.
\item We do not currently have a shorthand to refer to the $i$th row of a matrix, but we reserve the right to define one later.%will refer to the $i$th row  of the matrix $\mlmat{A}$ as $\mlvec{A}_{i, \star}$ (note that we use a capital letter since it is not a column vector, and thus should be thought of as a $1 \times d$ matrix).
\item We will refer to the element at row $i$, column $j$ of matrix $\mlmat{A}$ as $a_{i, j}$.
\ee



\subsection{Independent versus Dependent variables}
We will use `x' to refer to independent (i.e., input) variables and `y' to refer to dependent (i.e., output) variables.  For instance, when describing training data, we will always use `x' to refer to the input variables and `y' to refer to the output variable.


\end{document}
